\documentclass[usenames,dvipsnames,10pt]{beamer}
\usepackage[utf8]{inputenc}
\usepackage[T1]{fontenc}
\usepackage[german]{babel}
\usepackage{tikz}
\title{Planung meines Studiums mit MS Outlook}
\date[ISPN ’80]{EK/01 - \today}
\author{Florian Tietjen, Karl Döring, Anna Hoffmann, Timo Türk, Eric Antosch}

\usetheme{hawsx}
\setbeamertemplate{section in toc}{%
  
      \begin{tikzpicture}
        
        \fill[color=UmUBlue] (1,1) circle (0.2cm);
        \node[anchor=center] (xlab) at (1,1) {\color{white}\inserttocsectionnumber};
        \node[anchor=west] (ylab) at (1.2,0.97) {\color{UmUBlue}\sffamily\textsf{\inserttocsection}};
      \end{tikzpicture}
  }
\begin{document}

\begin{frame}
\titlepage
\end{frame}
\begin{frame}{Inhaltsverzeichnis}
    \tableofcontents
\end{frame}
\framecard{ÜBERSICHT}
\section{Übersicht}

\begin{frame} 
    
    \frametitle{Übersicht} 
    \framesubtitle{Was ist MS Outlook?}
    \begin{exampleblock}{MS Outlook}
        MS Outlook ist ein E-Mail- und Kalendarprogramm von Microsoft und steht,
        wie alle Programme der Office Suite, für Käufer- und Abonnenten der Office-Produkte
        zur Verfügung.
    \end{exampleblock} 
    
\end{frame}

\begin{frame} 
    
    \frametitle{Übersicht} 
    \framesubtitle{Wie bekomme ich das Programm?}
    \begin{exampleblock}{Download und Installation}
        MS Outlook steht als Teil des Abonnement der HAW des Office 365-Angebots
        für jeden Studenten der HAW zur Verfügung. Nach dem Download des Programms von 
        der Microsoft-Webseite, meldet man sich nun über die HAW-Kennung an. 
    \end{exampleblock} 
    
\end{frame}


\begin{frame} 
    
    \frametitle{Übersicht} 
    \framesubtitle{Welche Funktionen erwarten mich?}
    \begin{block}{E-Mail}
        \begin{enumerate}
            \item Verschiedene Konten auf einem Client
            \item Offline-Zugriff auf bereits heruntergeladene Emails
            \item Benachrichtigungen bei Erhalt von Emails
            \item Kategorien für bestimmte Arten von Mails
        \end{enumerate}
    \end{block} 
    
\end{frame}
\begin{frame} 
    
    \frametitle{Übersicht} 
    \framesubtitle{Welche Funktionen erwarten mich?}
    \begin{block}{Kalendar}
        \begin{enumerate}
            \item Mehrere Kalendar für verschiedene Anlässe
            \item Offline-Zugriff auf erstellte Pläne
            \item Benachrichtigungen für Events im Voraus
            \item Einfärbung für die Übersicht
            \item Synchronisierung der erstellten Kalendar über Geräte hinweg
        \end{enumerate}
    \end{block} 
    
\end{frame}
\section{Der Kalendar}
\framecard{Der Kalendar}

\framepic{calendar1.png}{
	%\framefill
    \textcolor{white}{}
    \vskip 0.5cm
}

\framepic{Calendar2.png}{
	%\framefill
    \textcolor{white}{}
    \vskip 0.5cm
}

\section{Interaktion mit Kommilitonen}
\framecard{Interaktion mit Kommilitonen}

\end{document}