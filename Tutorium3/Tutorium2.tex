\documentclass[usenames,dvipsnames,10pt]{beamer}
\usepackage[utf8]{inputenc}
\usepackage[T1]{fontenc}
\usepackage[german]{babel}
\usepackage{tikz}

%%%----------%%%----------%%%----------%%%----------%%%
\usepackage{listings}
\lstset{
    language=C,
    basicstyle=\ttfamily\small,
    aboveskip={1.0\baselineskip},
    belowskip={1.0\baselineskip},
    columns=fixed,
    extendedchars=true,
    breaklines=true,
    tabsize=4,
    prebreak=\raisebox{0ex}[0ex][0ex]{\ensuremath{\hookleftarrow}},
    frame=lines,
    showtabs=false,
    showspaces=false,
    showstringspaces=false,
    keywordstyle=\color[rgb]{0.627,0.126,0.941},
    commentstyle=\color[rgb]{0.133,0.545,0.133},
    stringstyle=\color[rgb]{01,0,0},
    numbers=left,
    numberstyle=\small,
    stepnumber=1,
    numbersep=10pt,
    captionpos=t,
    escapeinside={\%*}{*}
}

%%%----------%%%----------%%%----------%%%----------%%%
\title{Tutorium 3 - Programmieren 1}
\date[ISPN ’80]{PR/01 - \today}
\author{Eric Antosch}

\usetheme{hawsx}
\setbeamertemplate{section in toc}{%
  
      \begin{tikzpicture}
        
        \fill[color=UmUBlue] (1,1) circle (0.2cm);
        \node[anchor=center] (xlab) at (1,1) {\color{white}\inserttocsectionnumber};
        \node[anchor=west] (ylab) at (1.2,0.97) {\color{UmUBlue}\sffamily\textsf{\inserttocsection}};
      \end{tikzpicture}
  }


\begin{document}
    
    \begin{frame}
        \titlepage
    \end{frame}    
    \section{Aufgaben}
    \begin{frame}{Aufgaben}
        \begin{block}{Boxen durch einen Tunnel}
          Sie transportieren einige Boxen durch einen Tunnel, wobei jede Box durch ihre Länge, ihre Breite und Höhe definiert wird. Schreiben Sie zwei Funktionen, getVolume(int width, int height, int length) und fit(int height), wobei die erste ihnen das Volumen der Box zurückgibt und die zweite ihnen bestimmt, ob die Box durch den Tunnel passt. Eine Box passt, wenn ihre Höhe weniger als 41 ist.
        \end{block}
    \end{frame}
    \begin{frame}{Aufgaben}
        \begin{block}{Muster durch Schleifen}
          Erstellen Sie ein Muster aus den Zahlen 1 bis n, wobei n eine Zahl ist, die der User eingeben kann.
          Ein Beispiel für ein Muster mit n=4 wäre:
          \begin{center}
            4444444\\
            4333334\\
            4322234\\
            4321234\\
            4322234\\
            4333334\\
            4444444
          \end{center}
        \end{block}
    \end{frame}

    \begin{frame}{Aufgaben}
      \begin{block}{Plus Minus}
        Schreiben Sie ein Programm, welches eine Reihe von 5 Werten bekommt und von diesen dann entscheidet, ob die Menge eher negativ, positiv oder neutral ist. Geben Sie die Ergebnisse entweder in Prozent oder als Faktor an.
        Das jeweilige Ergebnis des Vergleiches schreiben sie dann darunter.
        \end{block}
      \end{frame}


    \begin{frame}{Aufgaben}
        \begin{block}{Äpfel und Orangen}
          Tom hat zwei Bäume auf seinem Grundstück, einen Apfel- und einen Orangenbaum. Sein Haus hat die Länge $s-t$, wobei s der Startpunkt und t der Endpunkt seines Hauses ist. Der Apfelbaum steht auf Punkt a und der Orangenbaum steht auf Punkt b, wobei $a<s$ und $b>t$. Mit gegebenen Werten a, b, s, t lässt sich somit ein Koordinatensystem mit den entsprechenden Bereichen darstellen. Von jedem Baum fallen pro Tag drei Äpfel und vier Orangen mit einer Distanz von d runter. Alle Werte von d sind im Bezug auf den Nullpunkt des Koordinatensystems angegeben. Schreiben Sie eine Funktion int drop(int a, int b, int s, int t, int d), die entscheidet, ob eine Frucht auf Toms Haus landet.
        \end{block}
    \end{frame}

    \begin{frame}{Aufgaben}
        \begin{block}{The Day of the Programmer}
          Marie hat eine Zeitmaschine entwickelt und möchte diese gerne testen, indem Sie zu dem Day of the Programmer (der 256te Tag des Jahres) in Russland reist. Bis zu dem Jahr 1918 galt in Russland noch der Julianische Kalender, in dem ein Schaltjahr immer nur durch seine Teilbarkeit durch 4 charakterisiert wurde. Alle Jahre danach sind wie bei uns dem Gregorianischem Kalender entsprechend gestaffelt, wobei ein Schaltjahr immer durch 4 und nicht durch 100 oder durch 400 teilbar sein muss. Schreiben Sie eine Funktion theDayOfTheProgrammer(int year), die ihnen den genauen Tag im September in diesem Jahr zurück gibt.
        \end{block}
    \end{frame}
    \begin{frame}{Aufgaben}
        \begin{block}{Utopia-Baum}
          Ein Utopia-Baum-Setzling wurde auf einer Grünfläche gepflanzt. Pro Jahr durchlebt ein Utopia-Baum zwei Wachstumszyklen, wobei der Baum im Frühling seine Größe verdoppelt und im Sommer genau um einen Meter wächst. Schreiben Sie eine Funktion growth(int n), die ihnen die Größe des Baumes nach n Wachstumszyklen ausgibt. Können Sie es schaffen, die ganze Liste auf der Konsole auszugeben?
        \end{block}
    \end{frame}
    \section{Mini-Praktikum}
    \begin{frame}{Mini-Praktikum}
        \begin{exampleblock}{Mini-Praktikum}
            Erstellen Sie ein Programm, welches die folgenden Anforderungen erfüllt:
            \begin{itemize}
                \item Sie haben eine Main-Funktion, die sich als aller erste Funktion in ihren Programm befindet.
                \item Sie haben zwei weitere Funktionen:
                \begin{itemize}
                    \item Eine Funktion gibt Ihnen die Summe der Quadratzahlen bis zu einer Zahl n zurück.
                    \item Die zweite Funktion gibt Ihnen das n.te (oder auch m.te) Glied der Fibonacci-Folge aus.
                \end{itemize}
                \item Beachten Sie, dass es manchmal sinnvoll sein kann, die Art, Funktionen zu schreiben, auf das Prinzip der Funktion anzupassen!
            \end{itemize}
        \end{exampleblock}
    \end{frame}
    

\end{document}
