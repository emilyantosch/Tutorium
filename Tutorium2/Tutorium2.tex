\documentclass[usenames,dvipsnames,10pt]{beamer}
\usepackage[utf8]{inputenc}
\usepackage[T1]{fontenc}
\usepackage[german]{babel}
\usepackage{tikz}

%%%----------%%%----------%%%----------%%%----------%%%
\usepackage{listings}
\lstset{
    language=C,
    basicstyle=\ttfamily\small,
    aboveskip={1.0\baselineskip},
    belowskip={1.0\baselineskip},
    columns=fixed,
    extendedchars=true,
    breaklines=true,
    tabsize=4,
    prebreak=\raisebox{0ex}[0ex][0ex]{\ensuremath{\hookleftarrow}},
    frame=lines,
    showtabs=false,
    showspaces=false,
    showstringspaces=false,
    keywordstyle=\color[rgb]{0.627,0.126,0.941},
    commentstyle=\color[rgb]{0.133,0.545,0.133},
    stringstyle=\color[rgb]{01,0,0},
    numbers=left,
    numberstyle=\small,
    stepnumber=1,
    numbersep=10pt,
    captionpos=t,
    escapeinside={\%*}{*}
}

%%%----------%%%----------%%%----------%%%----------%%%
\title{Tutorium 2 - Programmieren 1}
\date[ISPN ’80]{PR/01 - \today}
\author{Eric Antosch}

\usetheme{hawsx}
\setbeamertemplate{section in toc}{%
  
      \begin{tikzpicture}
        
        \fill[color=UmUBlue] (1,1) circle (0.2cm);
        \node[anchor=center] (xlab) at (1,1) {\color{white}\inserttocsectionnumber};
        \node[anchor=west] (ylab) at (1.2,0.97) {\color{UmUBlue}\sffamily\textsf{\inserttocsection}};
      \end{tikzpicture}
  }


\begin{document}
    
    \begin{frame}
        \titlepage
    \end{frame}    
    \begin{frame}{Inhaltsverzeichnis}
        \tableofcontents
    \end{frame}
    \section{Zusammenfassung}
    \begin{frame}{Zusammenfassung}
        \begin{block}{Sie haben schon gelernt...}
            \begin{itemize}
                \item was Funktionen sind und wie man sie definiert,
                \item was Headerdateien sind, welche es gibt und wie man eigene erstellt,
                \item wo der Unterschied zwischen Call-By-Value und Call-By-Reference ist
                \item und was Rekursion ist und wie man diese erstellt.
            \end{itemize}
        \end{block}
    \end{frame}
    \subsection{Funktion}
    \begin{frame}{Funktionen}
        \begin{block}{Was ist eine Funktion?}
            Eine Funktion ist ein Block von Code, der durch einen Aufruf ausgeführt werden kann. Dabei besteht jede Funktion aus einem 
            Funktionskopf und einem Funktionskörper. Im Funktionskopf wird der Name, der Typ des Rückgabewerts und die Parameter festgelegt, während
            im Funktionskörper der eigentliche Code vorgefunden werden kann.
        \end{block}
    \end{frame}
    \begin{frame}{Funktionen}{Prototypen}
        \begin{block}{Prototypen}
            Es ist ratsam, alle Funktionen in dem eigenen Code direkt unter den Includes und den Präprozessordirektiven mithilfe von Prototypen
            zu beschreiben. Der Prototyp ist im Prinzip nur der Funktionskopf ohne Funktionskörper.
        \end{block}
    \end{frame}
    \begin{frame}[fragile]{Funktionen}{Einfaches Beispiel: main()}
           \begin{lstlisting}
               int main(void){
                   return 0; 
               }
           \end{lstlisting} 
    \end{frame}
    \begin{frame}[fragile]{Funktionen}{Einfaches Beispiel: plus(a,b)}
        \begin{lstlisting}
            int plus(int a, int b){
                return a+b;
            }
        \end{lstlisting}
    \end{frame}
    \begin{frame}[fragile]
        \begin{lstlisting}
            
double decToBinary(int dec){

    int biggerPot = 2;
    int count = 0;
    while (dec%biggerPot!=dec)
    {
        biggerPot *= 2;     //biggerPot = biggerPot * 2;
        count++;
    }
    biggerPot /= 2;
    int pointOfIncrease = pow(10, count);
    double binary = 0;
    short firstTime = 1;
    while (count>-1)
    {
        int oldDec = dec;
        dec = dec%biggerPot;

    \end{lstlisting}
    \end{frame}
    \begin{frame}[fragile]
        \begin{lstlisting}
        if (firstTime == 1)
        {
            binary += pointOfIncrease;
            firstTime = 0;
        }else if (oldDec%biggerPot==oldDec)
        {
            
        }else
        {
            binary += pointOfIncrease;
        }
        
        
        
        biggerPot /= 2; // biggerPot = biggerPot / 2;
        count--;
        pointOfIncrease = pow(10, count);
        
    }
    return binary;
}
        \end{lstlisting}
    \end{frame}
    \subsection{Headerdateien}
    \begin{frame}{Headerdateien?}
        \begin{block}{Was sind Headerdateien?}
            Headerdateien sind der zweite Dateientyp neben *.c-Dateien, die ansich
            jedoch keinen C-Code enthalten. Sie enthalten lediglich die Informationen über tatsächliche
            C-Dateien, sodass sie in den eigenen Code eingebunden werden können.
        \end{block}
    \end{frame}

    \begin{frame}{Headerdateien}
        \begin{block}{Wichtige Headerdateien}
            Einige wichtige Headerdateien:
            \begin{itemize}
                \item stdio.h
                \item stdlib.h
                \item math.h
                \item conio.h
            \end{itemize}
        \end{block}
    \end{frame}
    \subsection{Value vs. Reference}
    \begin{frame}{Value vs. Reference}
        \begin{block}{Call-By-Value}
            Der Aufruf einer Funktion mit Parametern, die Variablen
            darstellen. Die Werte werden in der Funktion zur Verarbeitung 
            von Anweisungen und Informationen genutzt.
        \end{block}
    \end{frame}
    \begin{frame}[fragile]{Value vs. Reference}
        \begin{lstlisting}
            void tauschen(int a, int b){
                int temp = a;
                a = b;
                b = temp;
            }
        \end{lstlisting}
    \end{frame}
    \begin{frame}{Value vs. Reference}
        \begin{block}{Call-By-Reference}
            Der Aufruf einer Funktion mit Parametern, die Zeiger auf Variablen
            darstellen. Die Speicheraddressen, die den Variablen zugeordnet sind, werden in der Funktion zur Verarbeitung 
            von Anweisungen und Informationen genutzt. Es kann jedoch auch eine direkte Veränderung des Wertes aus der Funktion heraus geschehen.
        \end{block}
    \end{frame}
    \begin{frame}[fragile]{Value vs. Reference}
            \begin{lstlisting}
                void tauschen(int *a, int *b){
                    int temp = *a;
                    *a = *b;
                    *b = temp;
                }
            \end{lstlisting}
    \end{frame}
    \subsection{Rekursion}
    \begin{frame}{Rekursion}
        \begin{block}{Was ist Rekursion?}
            Als eine rekursive Funktion bezeichnet man eine Funktion, die sich in ihrer Ausführung selbst aufruft.
            Man kann solche Funktionen gut mit rekursiven Folgen vergleichen. Das nächste Folgenglied ist hierbei abhängig von 
            den vorherigen Ergebnissen der Folge. Ein Anfang wird dabei meist als gegeben angenommen.
        \end{block}
    \end{frame}
    \begin{frame}{Rekursion}
        \begin{block}{Stack}
            Zu beachten bei dem Erstellen von rekursiven Funktionen ist, dass ein sogenannter Stack existiert. Ruft eine Funktion sich selbst auf,
            dann wird zuerst die zuletzt aufgerufene Funktion bis zum Schluss ausgeführt. Nach dem Ende dieser Funktion, springt das Programm zurück 
            in die Funktion, die die gerade beendete Funktion aufgerufen hat. Rekursion ist also nicht einfach nur eine andere Form von Iteration.
        \end{block}
    \end{frame}
    \begin{frame}[fragile]{Rekursion}
        \begin{lstlisting}
            int addTillTen(int a){
                if(a >= 10){
                    return a;
                }
                a++;
                int b = addTillTen(a);
                printf("%d\n", b);
                return a;
            }
        \end{lstlisting}
    \end{frame}
    \section{Fragerunde}
\framepic{Idee.jpg}{
	%\framefill
    \textcolor{white}{Fragen?}
    \vskip 0.5cm
}

    \section{Aufgaben}
    \begin{frame}{Aufgaben}
        \begin{block}{Funktionen}
            Aus welchen zwei Teilen besteht eine Funktion in ihrem Code? Nennen Sie ein entsprechendes Beispiel für
            beide Teile!
        \end{block}
    \end{frame}
    \begin{frame}{Aufgaben}
        \begin{block}{Prototyp}
            Was ist der Prototyp einer Funktion und wofür wird er verwendet? Erstellen Sie eine Funktion und den entsprechenden Prototyp und 
            stellen Sie diesen kurz vor!
        \end{block}
    \end{frame}
    \begin{frame}{Aufgaben}
        \begin{block}{Headerdateien}
            Geben Sie in ihren eigenen Worten wieder, was eine Headerdatei ist! Welche Headerdateien sind schon bekannt?
            Geben Sie ein paar Beispiele für hilfreiche oder wichtige Funktionen aus diesen Dateien.
        \end{block}
    \end{frame}
    \begin{frame}{Aufgaben}
        \begin{block}{Call-By-Value}
            Erklären Sie kurz, was eine Funktion ist, die auf dem Call-By-Value-Prinzip beruht. Erstellen Sie schnell eine Funktion, die dieser Regel
            gehorcht. Wofür sind diese Funktionen besonders geeignet?
        \end{block}
    \end{frame}

    \begin{frame}{Aufgaben}
        \begin{block}{Call-By-Reference}
            Charakterisieren Sie nun das Call-By-Reference-Prinzip, indem Sie die Unterschiede zu Call-By-Value nennen und warum diese von Bedeutung sind.
            Schreiben Sie die gleiche Funktion aus der vorherigen Aufgabe nun auch mit Call-By-Reference. Ändert sich die Arbeitsweise ihrer Funktion und gibt es einen anderen Output? Wenn ja, warum?
        \end{block}
    \end{frame}
    \begin{frame}{Aufgaben}
        \begin{block}{Rekursion}
            Was ist Rekursion und woran erkennt man Rekursion in einem Programm? Nennen Sie Beispiele von Rekursion aus der Mathematik.
            Schreiben Sie eine Funktion, die über Rekursion funktioniert. Wo liegt der Unterschied zwischen Iteration und Rekursion?
        \end{block}
    \end{frame}
    \begin{frame}{Aufgaben}
        \begin{block}{Maximum von vier Zahlen}
            Schreiben Sie eine Funktion int maxOfFour(int a, int b, int c, int d), welche Ihnen das Maximum der vier Zahlen
            mithilfe von return wieder an ihre Main-Funktion schickt. Geben Sie dann diesen Wert mithilfe von printf in ihrer Main-Funktion aus.
        \end{block}
    \end{frame}
    \begin{frame}{Aufgaben}
        \begin{block}{Quersumme einer fünfstelligen Zahl}
            Schreiben Sie eine Funktion int quersumme(int n), die Ihnen die Quersumme einer fünfstelligen Zahl zurückgibt und geben Sie diese
            dann wieder in Ihrer Main-Funktion aus.
        \end{block}
    \end{frame}
    \section{Mini-Praktikum}
    \begin{frame}{Mini-Praktikum}
        \begin{exampleblock}{Mini-Praktikum}
            Erstellen Sie ein Programm, welches die folgenden Anforderungen erfüllt:
            \begin{itemize}
                \item Sie haben eine Main-Funktion, die sich als aller erste Funktion in ihren Programm befindet.
                \item Sie haben zwei weitere Funktionen:
                \begin{itemize}
                    \item Eine Funktion gibt Ihnen die Summe der Quadratzahlen bis zu einer Zahl n zurück.
                    \item Die zweite Funktion gibt Ihnen das n.te (oder auch m.te) Glied der Fibonacci-Folge aus.
                \end{itemize}
                \item Beachten Sie, dass es manchmal sinnvoll sein kann, die Art, Funktionen zu schreiben, auf das Prinzip der Funktion anzupassen!
            \end{itemize}
        \end{exampleblock}
    \end{frame}
    

\end{document}